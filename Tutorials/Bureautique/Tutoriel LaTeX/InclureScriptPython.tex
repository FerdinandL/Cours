\documentclass[article,10pt]{amsart}

\usepackage{amsmath, latexsym,hyperref, amsfonts, amssymb, amsthm,
  amscd,graphicx}

\usepackage[T1]{fontenc}
\usepackage[latin1]{inputenc}
\usepackage{color}
\title{}
\usepackage[a4paper, left=2cm, right=2cm, top=2cm,
bottom=2cm]{geometry}

\setlength{\parindent}{0pt}
% Inserer dans votre fichier .tex les lignes suivantes pour integrer
% du code source python dans vos rapports 
\usepackage{listings}

\usepackage{mathrsfs}
\definecolor{darkgreen}{rgb}{0.4,0.7,0}
\definecolor{gray}{rgb}{0.5,0.5,0.5}
\definecolor{lightgray}{rgb}{0.75,0.95,0.95}
\definecolor{mediumgray}{rgb}{0.60,0.60,0.60}
\definecolor{mauve}{rgb}{0.58,0,0.7}
\definecolor{darkblue}{rgb}{0,0,0.8}

\lstset{frame=tb,%
  morekeywords={as}%
  backgroundcolor=\color{lightgray},%
  language=Python,%
  aboveskip=3mm,%
  belowskip=3mm,%
  showstringspaces=false,%
  columns=flexible,%
  basicstyle={\small\ttfamily},%
  numbers=none,%
  numberstyle=\tiny\color{mauve},%
  keywordstyle={\color{darkblue}},%
  commentstyle=\color{mediumgray},%
  stringstyle=\color{darkgreen},%
  breaklines=true,%
  breakatwhitespace=false%
  tabsize=3%
}

\newcommand{\src}[1]{{\upshape \lstinline[language=Python]{#1}}}
% Fin des lignes a inserer

\begin{document}


\hrule \vspace*{2pt}
\noindent 
\textbf{Math\'ematiques Appliqu\'ees}
\hfill \textbf{Rapha\"el Deswarte} \\
\textbf{MAP571 -- Enseignement d'approfondissement}    \hfill \textbf{Caroline Hillairet} \\
\textbf{TP Initiation Python}\hfill \textbf{Aldjia Mazari}

\vspace*{2pt}

\begin{center}
  \Large \textbf{ Inclure du code \textsc{python} dans \LaTeX}
\end{center}

\vspace*{10pt}
Introduction du code \textsc{python} q1boucleVSvectoriel.py
\lstinputlisting[language=Python]{q1boucleVSvectoriel.py}

\end{document}



